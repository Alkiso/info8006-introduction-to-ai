\documentclass[a4paper, 10pt]{article}

\usepackage{vmargin}

\setmarginsrb{2cm}{0cm}{2cm}{0,2cm}{1cm}{1,5cm}{1cm}{1,5cm}
%1 est la marge gauche
%2 est la marge en haut
%3 est la marge droite
%4 est la marge en bas
%5 fixe la hauteur de l'entête
%6 fixe la distance entre l'entête et le texte
%7 fixe la hauteur du pied de page
%8 fixe la distance entre le texte et le pied de page
%------------------------------Packages généraux------------------------------

\usepackage[english]{babel}
\usepackage[T1]{fontenc}
\usepackage{ae}
\usepackage[utf8]{inputenc}
\usepackage{scrextend}
\usepackage{hyperref}

%-------------------------Mathématiques------------------------------
\usepackage{amsmath}
\usepackage{amssymb}
\usepackage{amsthm}
\usepackage{amsfonts}
\usepackage{eucal}
\newcommand\independent{\protect\mathpalette{\protect\independenT}{\perp}}
\def\independenT#1#2{\mathrel{\rlap{$#1#2$}\mkern2mu{#1#2}}}
%-----------------------Codes et algorithmes--------------------------
\usepackage{algorithm}
\usepackage{algorithmic}
\usepackage{clrscode3e}

%------------------------------Graphics------------------------------

\usepackage{graphicx}
\usepackage{fancyhdr}
\usepackage{fancybox}
\usepackage{color}
\usepackage{pgf, tikz}
\usetikzlibrary{arrows, automata}
%\usepackage{slashbox}
%------------------------------Syntaxe------------------------------

\usepackage{listings}
\lstloadlanguages{Matlab}

\def\refmark#1{\hbox{$^{\ref{#1}}$}}
\DeclareSymbolFont{cmmathcal}{OMS}{cmsy}{m}{n} %Mathcal correcte
\DeclareSymbolFontAlphabet{\mathcal}{cmmathcal}

%------------------------------Inclure code MatLab------------------------------

\usepackage{listings}
\newcommand*\styleC{\fontsize{9}{10pt}\usefont{T1}{ptm}{m}{n}\selectfont }
\newcommand*\styleD{\fontsize{9}{10pt}\usefont{OT1}{pag}{m}{n}\selectfont }

%------------------Sub-sections--------%
\usepackage{titlesec}
\usepackage{hyperref}

\renewcommand\thesubsubsection{\alph{subsubsection}}

\titleclass{\subsubsubsection}{straight}[\subsubsection]

\newcounter{subsubsubsection}[subsubsection]
\renewcommand\thesubsubsubsection{\thesubsubsection.\arabic{subsubsubsection}}

\titleformat{\subsubsubsection}
  {\normalfont\normalsize\bfseries}{\thesubsubsubsection}{1em}{}
\titlespacing*{\subsubsubsection}
{0pt}{3.25ex plus 1ex minus .2ex}{1.5ex plus .2ex}


\makeatletter
% on fixe le langage utilisé
\lstset{language=matlab}
\edef\Motscle{emph={\lst@keywords}}
\expandafter\lstset\expandafter{%
  \Motscle}
\makeatother


\definecolor{Ggris}{rgb}{0.45,0.48,0.45}

\lstset{emphstyle=\rmfamily\color{blue}, % les mots réservés de matlab en bleu
basicstyle=\styleC,
keywordstyle=\ttfamily,
commentstyle=\color{Ggris}\styleD, % commentaire en gris
numberstyle=\tiny\color{red},
numbers=left,
numbersep=10pt,
lineskip=0.7pt,
showstringspaces=false}
%  % inclure le fichier source
\newcommand{\FSource}[1]{%
\lstinputlisting[texcl=true]{#1}
}

\usepackage[section]{placeins}

\let\cleardoublepage\clearpage

\usepackage{hyperref}
               
 \hypersetup{
    colorlinks = true,
    linkcolor=black,
    urlcolor = black
    }
%------------------------------Début du document------------------------------
\begin{document}
%------------------------------Page de garde------------------------------

  % \frontmatter
  %\tableofcontents
   \newpage
   \setcounter{page}{1}
   %%%%%%%%% TP 1 %%%%%%%%%%%
   \section{Constraint satisfaction problems (18/10/2018)}
   \subsection{Objectives}
   At the end of this exercise session you should be able to:
   \begin{itemize}
       \item Define a search problem as constraint satisfaction problem (CSP).
       \item Draw the constraint graph of a CSP.
       \item Define what is backtracking, and to be able to use it.
       \item Define and apply variable and value ordering.
       \item Understand the syntax and the semantic of propositional logic.
       \item Check the entailment of two propositions.
       \item Show how a SAT problem can be cast into a CSP.
       
   \end{itemize}
   \subsection{Exercises}
   \subsubsection{8 Queens}
   "\textit{The eight queens puzzle is the problem of placing eight chess queens on an $8\times8$ chessboard so that no two queens threaten each other. Thus, a solution requires that no two queens share the same row, column, or diagonal. The eight queens puzzle is an example of the more general n queens problem of placing n non-attacking queens on an $n\times n$ chessboard, for which solutions exist for all natural numbers n with the exception of n=2 and n=3.}"\footnote{\url{https://en.wikipedia.org/wiki/Eight\_queens\_puzzle}}
   Consider the case where the chess board size $n = 4$. Answer the following questions:
   \begin{itemize}
       \item Define the problem as a CSP (implicit constraint are allowed).
       \item Draw the graph of constraints from your CSP formulation.
       \item Use backtracking search to find a solution.
   \end{itemize}
    % \subsubsection{Arc Consistency}
    % Use the AC-3 algorithm to show that arc consistency can detect the inconsistency of the partial assignment {WA=green, V =red} for the problem shown in Figure 6.1.
    \subsubsection{Propositional logic}
     If he studies well, he will pass the exam. If he does not like the course, he will not study well. He passed the exam. Therefore he
    liked the course.
    Is the conclusion really a logical consequence of the facts?
 \begin{itemize}
     \item Represent each sentence by a propositional logic formula.
     \item Give the models of each formula.
     \item Does the last sentence follow logically from the four first?
     \item Answer the previous question through the CSP formulation of the corresponding SAT problem.
 \end{itemize}
   \subsection{Supplementary materials}
   \url{https://www.codeproject.com/Articles/34403/Sudoku-as-a-CSP}\\
   \url{http://ai.berkeley.edu/sections/section_2_mA5IBOWiF6cw3yoIh65hXTiBY6mPiD.pdf}
\end{document}
